\documentclass{beamer}
%%\usetheme{Warsaw-seahorse}
\usepackage{german}
\usepackage{ngerman}
\usepackage{hyperref}

\usetheme{Warsaw}
%\usecolortheme{seahorse}
%\usefonttheme{serif}
\useinnertheme{rectangles}
%\usepackage{bookman}
\setbeamercovered{transparent}

%\setbeamertemplate{navigation symbols}{}
%\setbeamertemplate{footline}{}
%\setbeamertemplate{headline}{}


\begin{document}

\title{Projektdokumentation im Modul Semantic Web\\Anzahl der Publikationen pro Mitarbiter der HTWK-Leipzig}   
\author{Marcel Kirbst} 
\date{\today}
%\logo{\includegraphics[scale=0.08]{logo-SF}}

\begin{frame}
\titlepage
\end{frame}

\begin{frame}
\frametitle{Inhaltsverzeichnis}\tableofcontents
\end{frame}

\section{Recherchefragestellung}
\begin{frame}\frametitle{Recherchefragestellung} 
Eine Liste aller Angestellten der HTWK-Leipzig, geordnet nach der Anzahl der bisherigen Publikationen im deutschsprachigen Raum.
\end{frame}


\begin{frame}\frametitle{Inhaltliche Interpretation} 
\begin{itemize}
\item erfassen alle Mitarbeiter der HTWK in semantischer Datenbank
\item erfassen aller Publikationen im deutschsprachigen Raum
\item semantische Verknüpfung dieser Daten
\end{itemize}
\end{frame}

\begin{frame}\frametitle{Umsetzung} 
\begin{itemize}
\item Virtuelle Maschine Oracle VirtualBox
\item Betriebssystem Linux-Distribution Canonical Kubuntu 14.04
\item Datenbank Virtuoso + OntoWiki
\end{itemize}
\end{frame}


\section{relevante Datenquellen}
\begin{frame}\frametitle{HTWK-Telefonverzeichnis} 
\begin{itemize}
\item aktuellste, öffentlich zugängliche Ressource die gefunden wurde
\item abrufbar unter \url{http://www.htwk-leipzig.de/de/hochschule/telefonverzeichnis/}
\item als JSON aufbereitete Daten von Herrn Henri Knochenhauer, B.Sc und Herrn Roy Meissner, B.Sc. verfügbar
\item abrufbar unter \url{http://141.57.21.45:8080/info/staff}
\end{itemize}
\end{frame}

\begin{frame}\frametitle{Normdaten der Deutschen Nationalbibliothek} 
\begin{itemize}
\item sind öffentlich zugänglich, halbjährlich aktualisiert, liegen direkt im RDF-Format vor
\item jedoch sehr groß, (GND.rdf ca.9GB, DNBTitel.rdf ca.12GB), teilweise inkonsistent
\item abrufbar unter \url{http://datendienst.dnb.de/cgi-bin/mabit.pl?userID=opendata&pass=opendata&cmd=login}
\end{itemize}
\end{frame}


\section{Extraktion relevanter Daten}
\begin{frame}\frametitle{Importierbare Aufbereitung der Daten erforderlich} 
\begin{itemize}
\item HTWK-Mitarbeiterdaten: JSON $\Rightarrow$ RDF
\item DNB-Daten passen nicht in die VM, Extraktion der Daten die HTWK-Mitarbeiter betreffen
\item Implementierung eines JAVA-Programms
\end{itemize}
\end{frame}



\section{Verlinkung der Ressourcen}

\begin{frame}\frametitle{Verlinkung der Ressourcen direkt ueber Nachname, Vorname} 
\begin{itemize}
\item Vorteil: Mitarbeiterdaten lassen sich direkt mit den DNB-Titeldaten verknüpfen
\item Nachteil: sehr viele falsche Ergebnisse aufgrund doppelter Namen
\end{itemize}
\end{frame}

\begin{frame}\frametitle{Verlinkung der Ressourcen ueber eindeutige AutorenID} 
\begin{itemize}
\item (theoretischer) Vorteil: AutorenID ist theoretisch eindeutig, es existiert theoretisch nur eine AutorenID pro HTWK-Mitarbeiter
\item Praxis: Daten (im Moment noch) inkosistent, teilweise mehrere AutorenIDs, jeweilis unterschiedliche Publikationen zugeordnet
\end{itemize}
\end{frame}


\section{Anfrage an die Forschungswissensbasis} 
\begin{frame}\frametitle{SPARQL-Anfrage} 
siehe code
\end{frame}

\begin{frame}\frametitle{Ergebnis SPARQL-Anfrage} 
siehe code
\end{frame}

\section{Interpretation und Zusammenfassung}
\begin{frame}\frametitle{Interpretation und Zusammenfassung}
\begin{itemize}
\item Aufbereitung der Daten viel zeitintensiver als erwartet
\item aufgrund teilweise inkonsistenter Daten noch keine tragfähigen Ergebnisse
\item ggf manuelles Aufbereiten der HTWK-Daten (Geburtsjahr, DNB-ID hinzufügen) sollte auch zu besseren Ergebnissen führen
\end{itemize}
\end{frame}
\end{document}
